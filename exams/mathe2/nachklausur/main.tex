\documentclass[a4paper, 11pt]{article}
%%%% START OF PREAMBLE
% Copyright (c) 2021 Frederik, Franz, Marvin
% Copyright (c) 2022 Linus, Benny, Marvin
\usepackage[a4paper, inner=1cm, outer=1cm, top=2cm, bottom=2cm, bindingoffset=0cm]{geometry}
\usepackage{amsmath,amsthm,amssymb,amsfonts}
\usepackage{mathrsfs}
\usepackage{braket}
\usepackage{enumitem}
\usepackage{csquotes}
\usepackage{colortbl}
\usepackage{environ}
\usepackage{graphicx,tikz,xcolor,color,float,titlesec}
\usepackage{pgfplots}
\usepackage{fancyhdr}
\usepackage{gauss}
\usepackage{polynom}
\usepackage{bm}
\usepackage[ngerman=ngerman-x-latest]{hyphsubst}
\usepackage[ngerman]{babel}
\usetikzlibrary{matrix,shapes,trees}
\usepgfplotslibrary{fillbetween}
\pgfplotsset{compat=1.18}
\setlength\parindent{0pt}
\definecolor{ochsblau}{RGB}{26,122,219}
\definecolor{ochsgelb}{RGB}{245,209,168}
\definecolor{ochsorange}{RGB}{243,92,43}
\renewcommand{\headrulewidth}{2pt}
\let\oldheadrule\headrule
\renewcommand{\headrule}{\color{ochsgelb}\oldheadrule}
\newcommand\bracketify[1]{\lbrack#1\rbrack}
\titleformat{\section}{\normalfont\large\bfseries\color{ochsblau}}{Aufgabe \thesection.\ }{0em}{\bracketify}
%%%% END OF PREAMBLE

%\usepackage{background}
%\backgroundsetup{
%  position=current page.east,
%  angle=-90,
%  nodeanchor=east,
%  vshift=-5mm,
%  hshift=1cm,
%  opacity=1,
%  scale=3,
%  contents=Entwurf
%}

\newcommand\namesnstuff{
    \pagestyle{fancy}
    \fancyhf{}
    \fancyhead[L,LO]{\textcolor{ochsblau}{\textbf{Name:}}}
    \fancyhead[C,CO]{\textcolor{ochsblau}{\textbf{MFI2 - SS22}}}
    \fancyhead[R,RO]{\textcolor{ochsblau}{\textbf{Nachklausur}}}
    \fancyfoot[C,CO]{\thepage}
    \setlength{\headheight}{13.6pt}
    
    \vspace{-2.0cm}
    \noindent\includegraphics[width=0.2\textwidth]{../ochs_logo.png}
    \begin{minipage}[b]{0.6\textwidth}
    	\centering
    	\textcolor{ochsblau}{\textbf{\Large Mathematik 2 für Informatik}}\\
    	\vspace{3mm}
    	Peter Ochs, Tobias Nordgauer\\
    	\vspace{3mm}
    	Sommersemester 2022
    \end{minipage}
    \noindent\includegraphics[width=0.2\textwidth]{../uni_logo.png}\\
    \begin{center}\textcolor{ochsblau}{\textbf{Nachklausur Gedächtnisprotokoll}}\end{center}
}

\newcommand\refiff[1]{\stackrel{\text{#1}}{\iff}}
\newcommand\refimp[1]{\stackrel{\text{#1}}{\implies}}
\newcommand\refeq[1]{\stackrel{\text{#1}}{=}}
\newcommand\refless[1]{\stackrel{\text{#1}}{<}}
\newcommand\refleads[1]{\stackrel{\text{#1}}{\leadsto}}

\newcommand\NR{\textbf{Nebenrechnung: }}
\newcommand\proposition{\textbf{Behauptung: }}
\newcommand\toprove{\textbf{Zu zeigen: }}
\newcommand\task{\textbf{Aufgabe: }}
\newcommand\defi{\textbf{Definition: }}

% denglish ftw
\newcommand\da{\text{ da }}
\newcommand\with{\text{ mit }}
\newcommand\und{\text{ und }}
\newcommand\oder{\text{ oder }}
\newcommand\for{\text{ für }}
\newcommand\when{\text{ wenn }}
\newcommand\sei{\text{ sei }}

\newcommand\Real{\mathrm{Re}} % Realteil
\newcommand\Imag{\mathrm{Im}} % Imaginärteil

\newcommand\N{\mathbb{N}}
\newcommand\R{\mathbb{R}}
\newcommand\Z{\mathbb{Z}}
\newcommand\C{\mathbb{C}}
\newcommand\Q{\mathbb{Q}}
\renewcommand\P{\mathbb{P}}
\renewcommand\O{\mathcal{O}}
\newcommand\pot{\mathcal{P}}

\newcommand\rank{\mathrm{rank}}
\newcommand\lin{\mathrm{Lin}}
\renewcommand\det{\mathrm{det}}
\renewcommand\dim{\mathrm{dim}}

\renewcommand\u{\boldsymbol{u}}
\renewcommand\v{\boldsymbol{v}}
\newcommand\w{\boldsymbol{w}}

% for vertical line in gmatrix
\usepackage{etoolbox}
\makeatletter
\patchcmd\g@matrix
 {\vbox\bgroup}
 {\vbox\bgroup\normalbaselines}
 {}{}
\makeatother
\newcommand{\gvline}{%
  \hspace{-\arraycolsep}%
  \strut\vrule
  \hspace{-\arraycolsep}%
}

% lol
\makeatletter
\renewenvironment{proof}[1][\proofname] {\par\pushQED{\qed}\normalfont\topsep6\p@\@plus6\p@\relax\trivlist\item[\hskip\labelsep\bfseries#1\@addpunct{.}]\ignorespaces}{\popQED\endtrivlist\@endpefalse}
\makeatother
\def\qedsymbol{\sc q.e.d.} % hmm?

\newenvironment{induktion}{\renewcommand*{\proofname}{Beweis durch vollständige Induktion}\begin{proof}$ $\newline}{\end{proof}}
\newcommand\IA[1]{\textbf{Induktionsanfang ($#1$):}}
\newcommand\IV[1]{\textbf{Induktionsvoraussetzung:} Die Behauptung gelte für ein beliebiges aber festes $#1$.\\}
\newcommand\IS[1]{\textbf{Induktionsschritt ($#1$):}}

\newenvironment{gegenbeweis}{\renewcommand*{\proofname}{Beweis durch Gegenbeweis}\begin{proof}}{\end{proof}}
\newenvironment{gegenbeispiel}{\renewcommand*{\proofname}{Beweis durch Gegenbeispiel}\begin{proof}}{\end{proof}}

\NewEnviron{splitty}{\begin{displaymath}\begin{split}\BODY\end{split}\end{displaymath}}
\newcolumntype{C}{>{$}c<{$}} % math-mode column

\DeclareMathOperator{\ggT}{ggT}

\newlist{abc}{enumerate}{10}
\setlist[abc]{label=(\alph*)}

\newlist{num}{enumerate}{10}
\setlist[num]{label=\arabic*.}

\newlist{rom}{enumerate}{10}
\setlist[rom]{label=(\roman*)}


\begin{document}
\namesnstuff

\textbf{Bedingungen: 120min Zeit, einseitig beschriebenes Cheat-Sheet}

\section{10 Punkte}
Finden Sie mithilfe des Chinesischen Restsatzes alle $x\in\Z$, sodass
\begin{align*}
    x&\equiv2\pmod{3}\\
    x&\equiv3\pmod{4}\\
    x&\equiv5\pmod{7}
\end{align*}
gilt.

\section{10 Punkte}
Betrachten Sie die Matrix $$A=\begin{pmatrix}1&7&0&-1\\0&1&14&6\\-1&-2&1&-1\\4&-7&0&2\end{pmatrix}\in(\Z/7\Z)^{4\times4}$$
Bestimmen Sie, falls existent, die Inverse von $A$. Geben Sie die Einträge von $A^{-1}$ mit den kanonischen Repräsentaten $\{0,1,2,3,4,5,6\}$ aus $\Z/7\Z$ an.

\textit{Beachten Sie:} Wie üblich sind die Zahlen als Restklassen zu lesen.

\section{4 + 3 + 3 = 10 Punkte}
Betrachten Sie die Matrix $$A=\begin{pmatrix}5&3&-3\\0&-1&0\\6&3&-4\end{pmatrix}\in\R^{3\times3}.$$
\begin{abc}
    \item Bestimmen Sie alle Eigenwerte von $A$ und die zugehörigen Eigenräume.
    \item Entscheiden Sie über Diagonalisierbarkeit von $A$ und geben Sie gegebenenfalls eine Diagonalmatrix $D$ und eine invertierbare Matrix $S$ an, sodass $S^{-1}AS = D$.
    \item Bestimmen Sie $A^n$ für $n=10$.
\end{abc}

\section{3 + 4 + 3 = 10 Punkte}
Betrachten Sie für $I := \begin{pmatrix}0&1\\-1&0\end{pmatrix}$ und $E_2 := \begin{pmatrix}1&0\\0&1\end{pmatrix}\in\R^{2\times2}$ die Menge $$V := \{\lambda\cdot E_2 + \mu \cdot I\mid\lambda,\mu\in\R\}$$
\begin{abc}
    \item Zeigen Sie: $V$ ist ein $\R$-Untervektorraum von $\R^{2\times2}$.
    \item Bestimmen Sie eine Basis $\mathcal{B}$ von $V$ als $\R$-Vektorraum und folgern Sie die Dimension von $V$.
    \item Betrachten Sie nun den Vektorraum $U:=\R[X]_{\le 3}$ der Polynome mit Koeffizienten aus $\R$ vom Grad $\le3$ und die Basis $\mathcal{A}:=(X^0,X,X^2,X^3)$ (dass $\mathcal{A}$ eine Basis von $U$ ist, muss nicht gezeigt werden). Sei außerdem $$\varphi: U\to V,\quad p(X)\mapsto p(I)$$ die Abbildung, die $I$ in ein Polynom aus $U$ anstelle der Unbekannten $X$ einsetzt (dabei ist $I^0$ definiert als die Einheitsmatrix $E_2$). Diese Abbildung ist linear und wohldefiniert (muss nicht gezeigt werden). Bestimmen Sie die Darstellungsmatrix $M_\mathcal{A}^\mathcal{B}(\varphi)$ von $\varphi$ bezüglich der Basen $\mathcal{A}$ und $\mathcal{B}$.
\end{abc}

\section{3 + 3 + 4 = 10 Punkte}
Sei $\varphi: V\to V$ ein Isomorphismus zwischen zwei $K$-Vektorräumen.
\begin{abc}
    \item Zeigen Sie: $0$ ist kein Eigenwert von $\varphi$.
    \item Zeigen Sie: Ist $\lambda$ Eigenwert von $\varphi$, so ist $\lambda^{-1}$ Eigenwert von $\varphi^{-1}$.
    \item Sei nun $(V,\braket{\cdot,\cdot})$ ein reeler Prä-Hilbertraum und $\phi: V\to V$ eine orthogonale Abbildung, d.h. es gilt für alle $v,w\in V$:
    $$\braket{\phi(v), \phi(w)} = \braket{v,w}$$ Zeigen Sie: Die einzigen möglichen Eigenwerte von $\phi$ sind $\pm1$.
\end{abc}

\section{2 + 2 + 2 + 2 + 2 = 10 Punkte}
Entscheiden Sie über folgende Aussagen, ob sie wahr oder falsch sind. Begründen Sie ihre Antwort.
\begin{abc}
    \item Für zwei Polynome $p,q\in\Z/20\Z[X]$ gilt stets: $\mathrm{grad}(pq)=\mathrm{grad}(p)+\mathrm{grad}(q)$.
    \item $X^2+4$ hat in $\Z/13\Z$ genau 2 Nullstellen.
    \item Es gibt ganze Zahlen $r,s\in\Z$, sodass $3=r\cdot42+s\cdot99$
    \item Jede lineare Abbildung $\varphi: \R^3\to\R^3$ ist ein Isomorphismus.
    \item Jede Gruppe, deren Ordnung eine Primzahl ist, ist zyklisch.
\end{abc}
\par\hrulefill\par
\begin{center}
    \textbf{Danke für die Hilfe an alle Beteiligten.\\Keine Garantie auf Korrektheit.\\\LaTeX\ von Marvin Borner.}
\end{center}
\end{document}
