%%%% START OF PREAMBLE
% Copyright (c) 2021 Frederik, Franz, Marvin
% Copyright (c) 2022 Linus, Benny, Marvin
\usepackage[a4paper, inner=1cm, outer=1cm, top=2cm, bottom=2cm, bindingoffset=0cm]{geometry}
\usepackage{amsmath,amsthm,amssymb,amsfonts}
\usepackage{mathrsfs}
\usepackage{braket}
\usepackage{enumitem}
\usepackage{csquotes}
\usepackage{colortbl}
\usepackage{environ}
\usepackage{graphicx,tikz,xcolor,color,float,titlesec}
\usepackage{pgfplots}
\usepackage{fancyhdr}
\usepackage{gauss}
\usepackage{polynom}
\usepackage{bm}
\usepackage[ngerman=ngerman-x-latest]{hyphsubst}
\usepackage[ngerman]{babel}
\usetikzlibrary{matrix,shapes,trees}
\usepgfplotslibrary{fillbetween}
\pgfplotsset{compat=1.18}
\setlength\parindent{0pt}
\definecolor{ochsblau}{RGB}{26,122,219}
\definecolor{ochsgelb}{RGB}{245,209,168}
\definecolor{ochsorange}{RGB}{243,92,43}
\renewcommand{\headrulewidth}{2pt}
\let\oldheadrule\headrule
\renewcommand{\headrule}{\color{ochsgelb}\oldheadrule}
\newcommand\bracketify[1]{\lbrack#1\rbrack}
\titleformat{\section}{\normalfont\large\bfseries\color{ochsblau}}{Aufgabe \thesection.\ }{0em}{\bracketify}
%%%% END OF PREAMBLE

%\usepackage{background}
%\backgroundsetup{
%  position=current page.east,
%  angle=-90,
%  nodeanchor=east,
%  vshift=-5mm,
%  hshift=1cm,
%  opacity=1,
%  scale=3,
%  contents=Entwurf
%}

\newcommand\namesnstuff{
    \pagestyle{fancy}
    \fancyhf{}
    \fancyhead[L,LO]{\textcolor{ochsblau}{\textbf{Name:}}}
    \fancyhead[C,CO]{\textcolor{ochsblau}{\textbf{MFI2 - SS22}}}
    \fancyhead[R,RO]{\textcolor{ochsblau}{\textbf{Nachklausur}}}
    \fancyfoot[C,CO]{\thepage}
    \setlength{\headheight}{13.6pt}
    
    \vspace{-2.0cm}
    \noindent\includegraphics[width=0.2\textwidth]{../ochs_logo.png}
    \begin{minipage}[b]{0.6\textwidth}
    	\centering
    	\textcolor{ochsblau}{\textbf{\Large Mathematik 2 für Informatik}}\\
    	\vspace{3mm}
    	Peter Ochs, Tobias Nordgauer\\
    	\vspace{3mm}
    	Sommersemester 2022
    \end{minipage}
    \noindent\includegraphics[width=0.2\textwidth]{../uni_logo.png}\\
    \begin{center}\textcolor{ochsblau}{\textbf{Nachklausur Gedächtnisprotokoll}}\end{center}
}

\newcommand\refiff[1]{\stackrel{\text{#1}}{\iff}}
\newcommand\refimp[1]{\stackrel{\text{#1}}{\implies}}
\newcommand\refeq[1]{\stackrel{\text{#1}}{=}}
\newcommand\refless[1]{\stackrel{\text{#1}}{<}}
\newcommand\refleads[1]{\stackrel{\text{#1}}{\leadsto}}

\newcommand\NR{\textbf{Nebenrechnung: }}
\newcommand\proposition{\textbf{Behauptung: }}
\newcommand\toprove{\textbf{Zu zeigen: }}
\newcommand\task{\textbf{Aufgabe: }}
\newcommand\defi{\textbf{Definition: }}

% denglish ftw
\newcommand\da{\text{ da }}
\newcommand\with{\text{ mit }}
\newcommand\und{\text{ und }}
\newcommand\oder{\text{ oder }}
\newcommand\for{\text{ für }}
\newcommand\when{\text{ wenn }}
\newcommand\sei{\text{ sei }}

\newcommand\Real{\mathrm{Re}} % Realteil
\newcommand\Imag{\mathrm{Im}} % Imaginärteil

\newcommand\N{\mathbb{N}}
\newcommand\R{\mathbb{R}}
\newcommand\Z{\mathbb{Z}}
\newcommand\C{\mathbb{C}}
\newcommand\Q{\mathbb{Q}}
\renewcommand\P{\mathbb{P}}
\renewcommand\O{\mathcal{O}}
\newcommand\pot{\mathcal{P}}

\newcommand\rank{\mathrm{rank}}
\newcommand\lin{\mathrm{Lin}}
\renewcommand\det{\mathrm{det}}
\renewcommand\dim{\mathrm{dim}}

\renewcommand\u{\boldsymbol{u}}
\renewcommand\v{\boldsymbol{v}}
\newcommand\w{\boldsymbol{w}}

% for vertical line in gmatrix
\usepackage{etoolbox}
\makeatletter
\patchcmd\g@matrix
 {\vbox\bgroup}
 {\vbox\bgroup\normalbaselines}
 {}{}
\makeatother
\newcommand{\gvline}{%
  \hspace{-\arraycolsep}%
  \strut\vrule
  \hspace{-\arraycolsep}%
}

% lol
\makeatletter
\renewenvironment{proof}[1][\proofname] {\par\pushQED{\qed}\normalfont\topsep6\p@\@plus6\p@\relax\trivlist\item[\hskip\labelsep\bfseries#1\@addpunct{.}]\ignorespaces}{\popQED\endtrivlist\@endpefalse}
\makeatother
\def\qedsymbol{\sc q.e.d.} % hmm?

\newenvironment{induktion}{\renewcommand*{\proofname}{Beweis durch vollständige Induktion}\begin{proof}$ $\newline}{\end{proof}}
\newcommand\IA[1]{\textbf{Induktionsanfang ($#1$):}}
\newcommand\IV[1]{\textbf{Induktionsvoraussetzung:} Die Behauptung gelte für ein beliebiges aber festes $#1$.\\}
\newcommand\IS[1]{\textbf{Induktionsschritt ($#1$):}}

\newenvironment{gegenbeweis}{\renewcommand*{\proofname}{Beweis durch Gegenbeweis}\begin{proof}}{\end{proof}}
\newenvironment{gegenbeispiel}{\renewcommand*{\proofname}{Beweis durch Gegenbeispiel}\begin{proof}}{\end{proof}}

\NewEnviron{splitty}{\begin{displaymath}\begin{split}\BODY\end{split}\end{displaymath}}
\newcolumntype{C}{>{$}c<{$}} % math-mode column

\DeclareMathOperator{\ggT}{ggT}

\newlist{abc}{enumerate}{10}
\setlist[abc]{label=(\alph*)}

\newlist{num}{enumerate}{10}
\setlist[num]{label=\arabic*.}

\newlist{rom}{enumerate}{10}
\setlist[rom]{label=(\roman*)}

