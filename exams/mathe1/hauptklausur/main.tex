\documentclass[a4paper, 11pt]{article}

% Packages
\usepackage[a4paper, inner=2.5cm, outer=2.5cm, top=2.5cm, bottom=2.5cm, bindingoffset=0cm]{geometry}
\usepackage{amsmath,amsthm,amssymb,amsfonts}
\usepackage{graphicx}
\usepackage[colorlinks=true, allcolors=blue]{hyperref}
\usepackage{fontspec,xunicode,xltxtra}
\usepackage{biblatex}

% GERMAN
\usepackage[ngerman=ngerman-x-latest]{hyphsubst}
\usepackage[ngerman]{babel}

\usepackage{enumitem}

% Figures
\graphicspath{{figures/}}

\newcommand{\N}{\mathbb{N}}
\newcommand{\R}{\mathbb{R}}
\newcommand{\Z}{\mathbb{Z}}
\newcommand{\C}{\mathbb{C}}

\begin{document}

\title{\vspace{-2.0cm}Hauptklausur zu Mathematik 1 für Informatik}
\author{Peter Ochs, Oskar Adolfson}
\date{\today}

\maketitle

\begin{center}
	Hilfsmittel: Stift, einseitig beschriftetes DIN A4 Blatt.\\
	Zeit: 120min\\
	\textbf{Keine Garantie auf korrekte Aufgaben/Punktezahlen.}
\end{center}

\section*{Aufgabe 1 [3+2+5=10]}
Sei $z\in\C$.
\begin{enumerate}[label=(\alph*)]
	\item Berechnen Sie $z^8$ für $z=-1+i$ in der Form $z = a+ib$.
	\item Schreiben Sie $\frac{5}{i-2}$ in der Form $z = a+ib$.
	\item Berechnen Sie $z$ für $z^6=-64$.
\end{enumerate}

\section*{Aufgabe 2 [10]}
Es gelte $f(x) = \frac{sin(x)}{\sqrt{x}}$. Zeigen Sie, dass für alle $x\in(0,\infty)$ gilt $$f''(x)+\frac{1}{x} \cdot f'(x) + \left(1-\frac{1}{4x^2}\right) \cdot f(x) = 0.$$

\section*{Aufgabe 3 [2+3+2+3=10]}
Für die Folge $(a_n)_{n\in\N}$ gelte $a_0 = 1$ und $a_{n+1} = \sqrt{1+a_n}$.
\begin{enumerate}[label=(\alph*)]
	\item Zeigen Sie, dass $(a_n)_{n\in\N}$ monoton ist.
	\item Zeigen Sie, dass $(a_n)_{n\in\N}$ beschränkt ist.
	\item Zeigen Sie, dass $(a_n)_{n\in\N}$ konvergiert.
	\item Bestimmen Sie den Grenzwert von $(a_n)_{n\in\N}$.
\end{enumerate}

\section*{Aufgabe 4 [3+4+3=10]}
$g$ sei eine Folge von Funktionen mit $g_n = \frac{nx}{1+|nx|}$.
\begin{enumerate}[label=(\alph*)]
	\item Zeigen Sie, dass $g_n$ für alle $n\in\N$ stetig ist.
	\item Bestimmen Sie die Grenzfunktion von $g_n$.
	\item Zeigen Sie, dass die Folge nicht gleichmäßig konvergiert.
\end{enumerate}

\section*{Aufgabe 5 [3+4+3=10]}
Die Funktion $f$ in $\R$ sei zweifach stetig differenzierbar mit $f(0) = f'(0) = 0$ und $\forall x\in\R: f''(x) \ge 0$.
\begin{enumerate}[label=(\alph*)]
	\item Zeigen Sie, dass $\forall x\in\R: f(x) \ge 0$.
	\item Zeigen Sie, dass ein $c\in\R$ mit $c>1$ existiert, sodass für alle $k\in\R, k \ge 1$ gilt $$0 \le f\left(\frac{1}{k}\right) \le \frac{c}{k^2}.$$
	\item Zeigen Sie, dass die Reihe $\sum_{k=1}^\infty f\left(\frac{1}{k}\right)$ konvergiert.
\end{enumerate}

\section*{Aufgabe 6 [6+4=10]}
Eine Funktion $f$ heißt konvex, wenn gilt $$f(\lambda x + (1 - \lambda) y) \le \lambda f(x) + (1-\lambda)f(y)\quad\forall x,y\in\R,\lambda\in[0,1].$$
Eine Funktion $f$ heißt \textit{strikt} konvex, wenn gilt $$f(\lambda x + (1 - \lambda) y) < \lambda f(x) + (1-\lambda)f(y)\quad\forall x,y\in\R,x \ne y,\lambda\in[0,1].$$
\begin{enumerate}[label=(\alph*)]
	\item Zeigen Sie, dass für eine konvexe Funktion $f$ jedes lokale Minimum in $f$ auch das globale Minimum in $f$ ist.
	\item Zeigen Sie, dass für eine strikt konvexe Funktion $f$ sogar nur ein globales Minimum existiert.
\end{enumerate}

\noindent\rule[0.5ex]{\linewidth}{1pt}

\begin{center}
	Einschätzung: Schwierig.\\
	GeTeXt von Marvin Borner.
\end{center}

\end{document}
